غالب کتاب های درسی سیستم های عامل در دوره کارشناسی بخشی در همگام سازی دارند که به طور معمول شامل معرفی اجزای اولیه‌ای (موتکس، سمافور، ناظر و متغیر‌های شرطی) و مسائل کلاسیک مثل خواننده نویسنده و تولیدکننده مصرف کننده.
وقتی که من در برکلی کلاسی سیستم عامل را داشتم، و در کالج کالبی این درس را تدریس کردم، به این نتیجه رسیدم که بیشتر دانشجویان قادر به درک راه حل ارائه شده برای اینگونه مسائل هستند، اما تنها برخی از این دانشجویان توانایی [ارئه چنین راه حل‌هایی][ارئه همان راه‌حل‌ها] و  حل مسائل مشابه را دارند.

یکی از دلایلی که دانشجویان نمی توانند به طور عمیق این قبیل مسایل را بفهمند، این است که وقت و تلاش بیشتری می برند از آنچیزی که کلاس ها در اختیارشان می گذارد.همگام‌سازی یکی از ماژول‌هایی است که نسبت به دیگر ماژول‌ها وقت بیشتری نیاز دارد. و من مطمئن نیستم که بتوانم برای این منظور دلایلی را شرح دهم، منتها من فکر می کنم که سمافور‌ها یکی از چالشی‌ترین، جالب‌ترین و سرگرمی‌ترین بخش‌های سیستم عامل می باشد.
با هدف شناساندن  اصطلاحات والگوهای همگام‌سازی به گونه ای که به صورت مستقل قابل درک باشد و بتوان از آنها برای حل مسائل پیچیده استفاده نمود، اولین ویرایش این کتاب نوشتم.
نوشتن کدهمگام‌سازی چالش‌های مختص به خود را دارد زیرا که با افزایش تعداد اجزا و تعداد تعاملات به طور غیر قابل کنترلی افزایش می یابد.


با این وجود در بین راه حل هایی که دیدم، الگوهایی یافتم و حداقل برخی  ره‌یافت های روشمند درست برای ترکیب راه حل‌ها رسیدم.
شانسی این را داشتم که در زمانی که در کالج ویلسلی بودم، این کتاب را به همراه کتاب درسی استاندارد استفاده کردم و در زمان تدریس درس مبحث همگام سازی را به شکل موازی با درس تدریس می کردم. هر هفته به دانشجویان چند صفحه از کتاب را می دادم که با یک معما تمام می شد و گاهی اوقات یه راهنمایی مختصر. و به آنها توصیه می کردم که به راهنمایی نگاه نکننده مگر اینکه گیر افتاده باشند.
و همچنین ابزارهایی برای تست راه حل‌ها دادم، یه تخته مغناطیسی کوچک که می تونستن کدهاشون رو بنویسند و یک بسته آهنربا برای نمایش تردهای در حال اجرا.

نتیجه بسیار چشمگیر بود، هر چه زمان بیشتری در اختیار داشنجویان می گذاشتم، عمق فهمشون بیشتر می شد، مهمتر اینکه غالبشون قادر به حل بیشتر معماها بودند، و در برخی حالات همان راه حل های کلاسیک را می یافتند و یا راه حل جدیدی را ایجاد می کردند.
وقتی که رفتم کالج گام بعدی را با ایجاد کلاس فوق‌برنامه همگام سازی برداشتم، که در آن کلاس این کتاب تدریس می شد و همچنین پیاده سازی دستورات اولیه همگام‌سازی در زبان اسمبلی x86 و پاسیکس و پیتون.
دانشجویانی که این درس را گرفتند در یافتن خطاهای نسخه نخست کمک کردند و چندتا از آنها راه‌حل‌هایی بهتر از راه حل های من ارائه داند در پایان ترم از هر کدام انها خواستم که یک مسائله جدید با ترجیحا با یک راه‌حل بنویسند. از این مشارکت ها در نسخه دوم استفاده کردم.

بخش باقی مانده از پیشگفتار:
تغییر دیگر در نسخه دوم، نوع نگارش یا نحو آن است. بعد از آنی که نسخه اول را نوشتم، من زبان برنامه نویسی پی‌تون را که نتنها یکی از عالیترین زبان های برنامه نویسی است بلکه یک زبان بسیار شبیه به شبه‌کد است را یاد گرفتم. در نتیجه من از یک شبه کد شبیه به c در ویرایش نخست به یک شبه کد شبیه به زبان پی‌تون تغیر دادم. در حقیقت، من یک شبیه ساز نوشتم که بسیاری از راه حل های ارائه شده در این کتاب را می تواند اجرا کند. خواننده هایی هم که با زبان پی‌تون آشنا نیستند نیز (ان‌شالله) می توانند آن را درک کنند.در مواردی که از ویژگی‌های زبان پی‌تون استفاده کردم، نحو زبان پی‌تون و نحوه‌ی کار کد را شرح داده‌ام. امیدوارم این تغیر زمینه خوانا تر کردن کتاب را بوجود آورده باشد. صفحه بندی این کتاب ممکن است کمی عجیب به نظر برسد! اما صفحات خالی نیز خود یک روش سودمند است، بعد از هر معما، یک فضای خالی را تا شبه‌راهنمایی که در صفحه بعد است، گذاشته ام و بعد از آن یک صفحه خالی دیگر برای حل مساله تا صفحه نمایش راه حل نهایی. زمانی که من از این کتاب در کلاسم استفاده می کنم، برخی از صفحات را از کتاب جدا می کنم و دانشجواینم آنها را بعدا صحافی می کنند! این سیستم صفحه بندی امکان جداکردن معما را بدون صفحات مربوط به راهنمایی ها، محقق می کند. بعضی اوقات بخش مربوط به راهنمایی را تا می کنم و از دیده شدن آن جلوگیری می کنم تا دانشجویان خود به حل مساله پرداخته و در زمان مناسب راه حل را ببینند. اگر کتاب را به شکل تک صفحه چاپ کنید(به شکل یک رو سفید! زمانی که پولتان زیادی کرده باشد!مترجم) می توانید از چاپ صفحه های سفید خودداری کنید(ظاهرا نویسنده برای ایالت‌های اصفهان نشین آمریکا هستند!مترجم). 
این کتاب یک کتاب رایگان است، این بدین معنی است که هر شخصی می تواند آنرا بخواند، رونوشت برداری کند، اصلاح کند و حتی بازپخش کند و اینها به دلیل نوع لیسانس مورد استفاده برای این کتاب است.امیدوارم افراد این کتاب را مناسب و کارا ببینند، اما بیشتر از آن امیدوارم که آنها برای ادامه فرایند توسعه ایرادات و پیشنهادات خود و همینطور مطالب بیشتر خود را برایم ارسال کنند.
با تشکر
آلن دونی
نیدهام، ماساچوست
سه شنبه، ۱۲ خرداد ۱۳۸۳ 
